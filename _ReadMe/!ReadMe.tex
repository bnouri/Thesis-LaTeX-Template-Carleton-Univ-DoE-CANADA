\documentclass{article}
\RequirePackage[letterpaper, left=1in,top=1in,right=1in,bottom=1in]{geometry}
\usepackage[utf8]{inputenc}
\RequirePackage[]{hyperref,xcolor,fancyvrb,tabularx}
\hypersetup{colorlinks=true,linkcolor=blue,filecolor=magenta,urlcolor=blue}

\title{Notes on CU-Thesis}
\author{Behzad Nouri}
\date{\today}

\begin{document}
\begin{titlepage}
\maketitle
\tableofcontents
\pagenumbering{roman}
\end{titlepage}
\pagenumbering{arabic}

\section{Thesis (Official) Requirements}
Before and in the process of preparation of the thesis you may need to consult with:
\begin{itemize}
\item \href{https://gradstudents.carleton.ca/thesis-requirements/}{Thesis Requirements}
\item \href{https://gradstudents.carleton.ca/thesis-requirements/formatting-guidelines/}{Formatting Guidelines}
\item \href{https://gradstudents.carleton.ca/thesis-requirements/copyright/}{Copyright \& Intellectual Property}   {\color{red} $\Longleftarrow$ !!\textbf{Important}!!}
\item \href{https://gradstudents.carleton.ca/thesis-requirements/thesis-forms-templates/}{Thesis Forms, Templates, \& Policies}
\item \href{https://gradstudents.carleton.ca/thesis-requirements/thesis-checklist/}{Thesis Checklist}
\item \href{https://gradstudents.carleton.ca/thesis-requirements/pdfa-formatting/}{Converting to PDF/A Format}
\item \href{https://gradstudents.carleton.ca/thesis-requirements/electronic/}{Electronic Thesis Deposit} 
\end{itemize}

\section{Important to Note}
\begin{itemize}
\item In the FINAL submission No internal links is allowed in the thesis final-PDF. 
\item They can be disabled by commenting-out the hyperref[]{} in the  "SBNth.sty" file!
\item Keeping it is helpful in the writing process though!
\item The acknowledgments, preface, contributions, and dedication sections are not required, but the ABSTRACT section is.
\end{itemize}

\section{Sections in the thesis}
%%%%%%%%%%%%%%%%%%%%%%%%%%%%%%%%%%%%%%%%%%%%%%%%%%%%%%%%%%%%%%%%%%%%%%%%%%%%%%%%
The following sections will mainly be created automatically:\par
\begin{tabular}{llp{4cm}}
1)&  Title page,&         Will be generated automatically\\
2)&  Signature page,&     Do not worry about this Department will take care of it!\\
3)&  Abstract&            \textbf{Mandatory}\\
4)&  Dedication Page&     \textbf{Optional}\\
5)&  Acknowledgments&     \textbf{Optional}\\
6)&  Preface&             \textbf{Optional}\\
7)&  Table of Contents& \\
8)&  List of Tables& \\
9)&  List of Figures& \\
10)& List of Acronyms& \\
11)& List of Symbols& \\
12)& Chapter 1: Background and Preliminaries& \\
23)& Chapters 2, 3, 4, ... & \\
24)& Chapter x: Conclusions and Future Work& \\
25)& List of References& \\
26)& Appendix 1,...& \\
\end{tabular}
%%%%%%%%%%%%%%%%%%%%%%%%%%%%%%%%%%%%%%%%%%%%%%%%%%%%%%%%%%%%%%%%%%%%%%%%%%%%%%%%%%%


%%%%%%%%%%%%%%%%%%%%%%%%%%%%%%%%%%%%%%%%%%%%%%%%%%%%%%%%%%%%%%%%%%%%%%%%%%%%%%%%
\section{Options for \texttt{cuthesis.sty}}

\begin{verbatim}
%%%%%%%%%%%%%%%%%%%%%%%%%%%%%%%%%%%%%%%%%%%%%%%%%%%%%%%%%%%%%%%%%%%%%%%%%%%%%%%%
% This LaTeX package is used to create theses at Carleton University. 
% -- It has several options which are described below.  
% -- Multiple options can be included as a comma separated list.  
% -- See the examples for common uses.
%
% Options: 
% -------
%   manuscript,standard: this specifies which format of thesis you will
%     be creating.  Manuscript format has the references at the end of each
%     chapter, while standard format has one reference section for the
%     whole document.  standard is the default.
%     
%     N.B.: Do NOT use ``Manuscript'' for the final submission.
%           It may be useful if you are editing your thesis chapter-by-chapter
%   
%   phd,masters: this specifies whether this is a PhD. dissertation
%     or a masters thesis.  masters is the default.
%
%
%   1committee, 2committee, 3committee,4committee,5committee: this is the number
%     of people on your committee (in addition to the department chair), which 
%     determines how many signature lines are needed.  Remember that the chairman 
%     of the committee does not sign the thesis.  
%     -- 1committee is the default.
%     NOTE: DUE TO RECENT CHANGES (2019) TO THE THESIS HANDLING PROCESS in 
%           the Department of Electronics: the ``SIGNATURE PAGE'', including 
%           the committee, will be taken care of by department , so do NOT 
%           worry about this. Leave it to be the default value!
%
%   sequential,nonsequential: this specifies whether you want numbering
%     of figures, equations, and tables reset to 1 at the beginning of
%     each chapter (nonsequential), or if you want the numbers to
%     be sequential throughout the whole document.  The default is
%     nonsequential for Manuscript format and sequential for
%     Standard format.
%     NOTE: I found ``nonsequential'' preferable as it makes it easier 
%           to trance the figures ...
%           Since CU does not explicitly mentioned any standard for this option 
%           Following our preference and in the favor of legibility, we set this 
%           option to nonsequential. (Dec.2020, Behzad)
%
%
% Examples:
% ---------
%   For a standard format masters thesis:
%     \usepackage{cuthesis}
%   
%   NOTE: CARLETON DOES NOT USE THE WORD DISSERTATION!
%   For a manuscript format PhD. dissertation:
%     \usepackage[manuscript,phd]{cuthesis}
%
%   For a standard format PhD. dissertation with a four member committee:
%     \usepackage[phd,4committee]{cuthesis}
%
%   
% Notes: 
% ------ 
%   (1) The acknowledgements, preface, contributions, and dedication sections 
%   are not required, but the ABSTRACT section is.
%   
%   (2) By default the department named on the title page is ``
%   Department of Electronics'', but that can be changed by putting the
%   command:      \dept{My Department}
%   in the main .tex file before any of the chapters are included.
%
%   (3) bibliography:
%   (a) In compliance with the Carleton University Library's biblical Format
%   The ``ieeetran.bst'' (IEEE Transactions) bibliography style is used.
%   Generally, it is a part of standard MikTeX distribution, and we do not 
%   to include it in our files.
%   (b) It uses the same font-size (10pt/12pt) defined in \documentclass[]().
%%%%%%%%%%%%%%%%%%%%%%%%%%%%%%%%%%%%%%%%%%%%%%%%%%%%%%%%%%%%%%%%%%%%%%%%%%%%%%%%
\end{verbatim}

\end{document}